\documentclass[11pt]{article}

%================================================================
%================================================================
%
%                           Preamble
%
%================================================================
%================================================================

%======================================================
%
%                      Packages
%
%======================================================

\usepackage[margin=1in]{geometry}  % set the margins to 1in
\usepackage{graphicx}              % to include figures
\usepackage{amsmath}               % great math stuff
\usepackage{amsfonts}              % for blackboard bold, etc
\usepackage{amsthm}                % better theorem environments
\usepackage{titlesec}              % format section titles

\usepackage{soul}
\usepackage{mathrsfs}
\usepackage{enumerate}
\usepackage{multicol}
\usepackage[makeroom]{cancel}
\usepackage{xcolor}
%\usepackage[usenames,dvipsnames]{color}
\usepackage{tikz}
\usepackage{setspace}
\usepackage{pdfpages}
\usepackage{listings}
\usepackage{matlab-prettifier}
\usepackage{xspace}

\usepackage{amssymb}
\usepackage{parskip}
\usepackage{color}
\usepackage[hyphens]{url}
\usepackage{latexsym}
\usepackage{fancyhdr}
\usepackage{fancyvrb}
\usepackage{algpseudocode}
\usepackage{verbatim}
\usepackage{collectbox}
\usepackage{scrextend}
\usetikzlibrary{arrows,shapes}

\DeclareMathOperator{\id}{id}

%======================================================
%
%                   New Commands
%
%======================================================

\newcommand{\bd}[1]{\mathbf{#1}}  % for bolding symbols
\newcommand{\RR}{\mathbb{R}}      % for Real numbers
\newcommand{\ZZ}{\mathbb{Z}}      % for Integers
\newcommand{\col}[1]{\left[\begin{matrix} #1 \end{matrix} \right]}
\newcommand{\comb}[2]{\binom{#1^2 + #2^2}{#1+#2}}
\newcommand{\overfrac}[2]{\genfrac{}{}{0pt}{}{#1}{#2}}

\newcommand{\numdash}{\nobreakdash--}
\newcommand{\blank}[1]{\underline{\hspace{#1}}}
\newcommand{\N}{\ensuremath{\mathbb{N}}}
\newcommand{\Z}{\ensuremath{\mathbb{Z}}}
\newcommand{\Q}{\ensuremath{\mathbb{Q}}}
\newcommand{\R}{\ensuremath{\mathbb{R}}}
\newcommand{\C}{\ensuremath{\mathbb{C}}}
\newcommand{\B}{\ensuremath{\mathbb{B}}}
\newcommand{\T}{\ensuremath{\mathbb{T}}}
\newcommand{\Tau}{\ensuremath{\mathcal{T}}}
\newcommand{\HS}{\ensuremath{\mathcal{H}}}
\newcommand{\intom}{\ensuremath{\int_{\Omega}}}
\newcommand{\fa}{\ensuremath{\ \forall\ }}
\newcommand{\ex}{\ensuremath{\ \exists\ }}
\newcommand{\idty}{{\mathchoice {\rm 1\mskip-4mu l} {\rm 1\mskip-4mu l} %
    {\rm 1\mskip-4.5mu l} {\rm 1\mskip-5mu l}}}
\newcommand{\MATLAB}{\textsc{Matlab}\xspace}

\newtheorem{proposition}{Proposition}[section]
\newtheorem{lemma}[proposition]{Lemma}
\newtheorem{theorem}[proposition]{Theorem}
\newtheorem{corollary}[proposition]{Corollary}
\newtheorem{conjecture}[proposition]{Conjecture}
\theoremstyle{definition}
\newtheorem{definition}[proposition]{Definition}
\newtheorem{example}[proposition]{Example}
\theoremstyle{remark}
\newtheorem{remark}[proposition]{Remark}
\newtheorem{claim}[proposition]{Claim}
\newtheorem{notation}[proposition]{Notation}

\def\Xint#1{\mathchoice
	{\XXint\displaystyle\textstyle{#1}}%
	{\XXint\textstyle\scriptstyle{#1}}%
	{\XXint\scriptstyle\scriptscriptstyle{#1}}%
	{\XXint\scriptscriptstyle\scriptscriptstyle{#1}}%
	\!\int}
\def\XXint#1#2#3{{\setbox0=\hbox{$#1{#2#3}{\int}$ }
		\vcenter{\hbox{$#2#3$ }}\kern-.6\wd0}}
\def\ddashint{\Xint=}
\def\dashint{\Xint-}

\makeatletter
\newcommand{\mybox}{%
	\collectbox{%
		\setlength{\fboxsep}{1pt}%
		\fbox{\BOXCONTENT}%
	}%
}
\makeatother

\newcommand{\newquestion}{\hrulefill\vspace{-0.8\baselineskip}\\\null\hrulefill\vspace{-1.0\baselineskip}}
\newcommand{\newpart}{\vspace{-0.5\baselineskip}\hrulefill\vspace{-1.3\baselineskip}}

\DeclareMathOperator{\ran}{ran}
\DeclareMathOperator{\krnl}{ker}
\DeclareMathOperator{\dist}{dist}
\DeclareMathOperator{\image}{im}
\DeclareMathOperator{\supp}{supp}
\DeclareMathOperator{\vol}{vol}
\DeclareMathOperator{\spn}{span}
\DeclareMathOperator{\GL}{GL}
\DeclareMathOperator{\card}{card}
\DeclareMathOperator{\LCM}{LCM}
\DeclareMathOperator{\HCF}{HCF}

%\numberwithin{equation}{chapter}

%======================================================
%
%                   Format Specifications
%
%======================================================

\everymath{\displaystyle}
\setlength\parindent{0pt}
\titleformat{\section}{\normalfont}{\thesection}{}{}
\titleformat{\subsection}{\normalfont}{\thesubsection}{}{}
\theoremstyle{plain}

\lstset{
  numbers=left,
  numberstyle=\scriptsize,
  stepnumber=1,
  numbersep=8pt,
  showstringspaces=false,
  breaklines=true,
  frame=single
}

%================================================================
%================================================================
%
%                          Homework #
%
%================================================================
%================================================================
\begin{document}
  \begin{flushright}
    Mikhail Gaerlan\\
    23 January 2018\\
    MAT 226B Freund
  \end{flushright}
\vspace{-1.3\baselineskip}
\newquestion
%======================================================
%
%                    Problem 1
%
%======================================================
\section*{Problem 1}
Let $0\leq\alpha\leq1$ and $A_\alpha=\left[a_{jk}^\alpha\right]$. Then
\begin{equation*}
  ee^{T}=\left[
    \begin{array}{cccc}
      1 & 1 & \cdots & 1\\
      1 & 1 & \cdots & 1\\
      \vdots & \vdots & \ddots & \vdots\\
      1 & 1 & \cdots & 1
    \end{array}
  \right].
\end{equation*}
Thus, $a_{jk}^\alpha=\alpha a_{jk}+\frac{1}{n}\left(1-\alpha\right]$. Also, $\alpha_{jk}^\alpha\geq0$ since $0\leq\alpha\leq1$, and so
\begin{equation*}
  \sum_{k=1}^na_{jk}^\alpha=\sum_{k=1}^n\left(\alpha a_{jk}+\frac{1}{n}\left(1-\alpha\right)\right)=n\frac{1}{n}\left(1-\alpha\right)+\alpha\sum_{k=1}^na_{jk}=1-\alpha+\alpha=1.
\end{equation*}
Therefore, $A_\alpha$ is row-stochastic.

\newquestion
%======================================================
%
%                    Problem 2
%
%======================================================
\section*{Problem 2}

The PageRank vector $x$ is given by the formula $A^{T}x=x$ where $A$ is given by the following matrix:
\begin{equation*}
  A=\left[
    \begin{array}{cccccccccc}
    % 1 & 2 & 3 & 4 & 5 & 6 & 7 & 8 & 9 &10\\
      0 & 0 & 0 & 0 & 1/2 & 0 & 1/2 & 0 & 0 & 0\\ %1
      0 & 0 & 0 & 0 & 0 & 1/3 & 1/3 & 0 & 1/3 & 0\\ %2
      1/10 & 1/10 & 1/10 & 1/10 & 1/10 & 1/10 & 1/10 & 1/10 & 1/10 & 1/10\\ %3
      0 & 0 & 0 & 0 & 0 & 0 & 1 & 0 & 0 & 0\\ %4
      0 & 0 & 0 & 0 & 0 & 0 & 1/3 & 0 & 1/3 & 1/3\\ %5
      1/10 & 1/10 & 1/10 & 1/10 & 1/10 & 1/10 & 1/10 & 1/10 & 1/10 & 1/10\\ %6
      1/10 & 1/10 & 1/10 & 1/10 & 1/10 & 1/10 & 1/10 & 1/10 & 1/10 & 1/10\\ %7
      0 & 0 & 1/4 & 1/4 & 0 & 0 & 1/4 & 0 & 1/4 & 0\\ %8
      0 & 1/5 & 0 & 1/5 & 0 & 0 & 1/5 & 1/5 & 0 & 1/5\\ %9
      1/5 & 0 & 1/5 & 1/5 & 0 & 0 & 1/5 & 0 & 1/5 & 0   %10
    \end{array}
  \right].
\end{equation*}
The following \MATLAB code was used to solve for $x$ and rank the eigenvalues.
\lstinputlisting[style=Matlab-editor,basicstyle=\ttfamily\small]{homework1_2.m}
\begin{equation*}
  x=\left[\begin{array}{l}
\texttt{1.611685764370067e-01}\\
\texttt{1.785251923609917e-01}\\
\texttt{2.057998745272545e-01}\\
\texttt{2.717395180598953e-01}\\
\texttt{1.931698370468546e-01}\\
\texttt{1.720939462820155e-01}\\
\texttt{7.479616674742428e-01}\\
\texttt{1.785251923609919e-01}\\
\texttt{3.296982176632033e-01}\\
\texttt{2.429151380432770e-01}\\
\end{array}
\right]
\end{equation*}
Thus the ranking of the websites from most to least important is given by the row vector:
\begin{equation*}
  b = \left[\begin{array}{llllllllll}
\texttt{7}&\texttt{9}&\texttt{4}&\texttt{10}&\texttt{3}&\texttt{5}&\texttt{8}&\texttt{2}&\texttt{6}&\texttt{1}\\
\end{array}
\right].
\end{equation*}

\newquestion
%======================================================
%
%                    Problem 3
%
%======================================================
\section*{Problem 3}

\newpart
%--------------------------
%    Problem 3 Part A
%--------------------------
\subsection*{(a)}
Let $x\in\R^n$. Then,
\begin{eqnarray*}
  \left\Vert A^Tx\right\Vert_1&=&\sum_{k=1}^n\left\vert\sum_{j=1}^na_{jk}x_j\right\vert\\
                           &\leq&\sum_{k=1}^n\sum_{j=1}^n\left\vert a_{jk}\right\vert\left\vert x_j\right\vert\\
                           &=&\sum_{j=1}^n\left\vert x_j\right\vert\sum_{k=1}^n\left\vert a_{jk}\right\vert\\
                           &=&\sum_{j=1}^n\left\vert x_j\right\vert\\
                           &=&\left\Vert x\right\Vert_1
\end{eqnarray*}
Since, $\left\Vert A^Tx\right\Vert_1\leq\left\Vert x\right\Vert_1\;\forall\, x\in\R^n$, then if $\lambda$ is an eigenvalue of $A^T$, then $\left\vert\lambda\right\vert\leq1$. Since $A$ and $A^T$ have the same eigenvalues, then $\lambda$ is an eigenvalue of $A$ and $\left\vert\lambda\right\vert\leq1$.

\newpart
%--------------------------
%    Problem 3 Part B
%--------------------------
\subsection*{(b)}
Since $a_{jk},x_j^{(0)}\geq0$, then $x^{(i)}\geq0$ by closure of the nonnegative real numbers under addition and multiplication. Also, since $x^{(0)}\geq0$, then the inequality from 3.(a) becomes an equality, so $\sum_{j=1}^nx_j^{(i)}=\sum_{j=1}^nx_j^{(0)}\;\forall\, i\geq0$.

\newpart
%--------------------------
%    Problem 3 Part C
%--------------------------
\subsection*{(c)}
By induction, $x^{(i)}=X\Lambda^i X^{-1}x^{(0)}$ where $\Lambda, \Lambda^i$ are given by the following matrices:
\begin{equation*}
  \Lambda=\left[
    \begin{array}{cccc}
      \lambda_1 & 0 & \cdots & 0\\
      0 & \lambda_2 & \ddots & \vdots\\
      \vdots & \ddots & \ddots & 0\\
      0 & \cdots & 0 & \lambda_n
    \end{array}
  \right]\textrm{and }
  \Lambda^i=\left[
    \begin{array}{cccc}
      \lambda_1^i & 0 & \cdots & 0\\
      0 & \lambda_2^i & \ddots & \vdots\\
      \vdots & \ddots & \ddots & 0\\
      0 & \cdots & 0 & \lambda_n^i
    \end{array}
  \right]
\end{equation*}
Therefore,
\begin{eqnarray*}
  \lim_{i\to\infty}x^{(i)}&=&\lim_{i\to\infty}X\Lambda^iX^{-1}x^{(0)}\\
                          &=&X\left(\lim_{i\to\infty}\Lambda^i\right)X^{-1}x^{(0)}
\end{eqnarray*}
Since $\left\vert\lambda_j\right\vert<1\;\forall\,2\leq j\leq n$, then $\lim_{i\to\infty}\lambda_j^i=
\left\{\begin{array}{cr}
        1 & \textrm{if }j = 1\\
        0 & \textrm{if }j > 1
      \end{array}
    \right.$. Thus $\lim_{i\to\infty}\Lambda^i$ converges. Since $x^{i}\geq0\;\forall\,i$, then $x\geq0$ by the properties of limits. Also, since $\sum_{j=1}^nx_j^{(i)}=\sum_{j=1}^nx_j^{(0)}\;\forall\, i\geq0$, then $x^{(0)}\ne0\implies x\ne0$.

\newquestion
%======================================================
%
%                    Problem 4
%
%======================================================
\section*{Problem 4}

\newpart
%--------------------------
%    Problem 4 Part A
%--------------------------
\subsection*{(a)}
For the case $n=10$ from Problem 2, the code
\lstinputlisting[style=Matlab-editor,basicstyle=\ttfamily\small]{homework1_4_a_1.m}\newpage
where the functions \texttt{graphx} and \texttt{graphax} are defined as the following
\lstinputlisting[style=Matlab-editor,basicstyle=\ttfamily\small]{graphx.m}
\lstinputlisting[style=Matlab-editor,basicstyle=\ttfamily\small]{graphax.m}
has the following output
\begin{equation*}
  \begin{array}{rclcrcl}
    y&=&\left[\begin{array}{l}
\texttt{5.000000000000000e-01}\\
\texttt{5.000000000000000e-01}\\
\texttt{7.500000000000000e-01}\\
\texttt{9.500000000000000e-01}\\
\texttt{8.000000000000000e-01}\\
\texttt{6.333333333333333e-01}\\
\texttt{3.116666666666666e+00}\\
\texttt{5.000000000000000e-01}\\
\texttt{1.416666666666667e+00}\\
\texttt{8.333333333333333e-01}\\
\end{array}
\right]\\
    y_{0.5}&=&\left[\begin{array}{l}
\texttt{7.500000000000000e-01}\\
\texttt{7.500000000000000e-01}\\
\texttt{8.750000000000000e-01}\\
\texttt{9.750000000000000e-01}\\
\texttt{9.000000000000000e-01}\\
\texttt{8.166666666666667e-01}\\
\texttt{2.058333333333334e+00}\\
\texttt{7.500000000000000e-01}\\
\texttt{1.208333333333333e+00}\\
\texttt{9.166666666666666e-01}\\
\end{array}
\right]&&y_{0.85}&=&\left[\begin{array}{l}
\texttt{5.750000000000001e-01}\\
\texttt{5.750000000000001e-01}\\
\texttt{7.875000000000000e-01}\\
\texttt{9.575000000000000e-01}\\
\texttt{8.300000000000000e-01}\\
\texttt{6.883333333333334e-01}\\
\texttt{2.799166666666666e+00}\\
\texttt{5.750000000000001e-01}\\
\texttt{1.354166666666667e+00}\\
\texttt{8.583333333333334e-01}\\
\end{array}
\right]
  \end{array}
\end{equation*}
For the case $n=685230$, the following code
\lstinputlisting[style=Matlab-editor,basicstyle=\ttfamily\small]{homework1_4_a.m}
resulted in the output
\begin{equation*}
  \begin{array}{c@{\hspace{0cm}}c@{\hspace{0cm}}lc@{\hspace{0cm}}c@{\hspace{0cm}}l}
    \left[
    \begin{array}{c}
      y_2\\
      y_{222222}\\
      y_{300000}\\
      y_{400000}\\
    \end{array}
    \right]&=&\left[\begin{array}{r}
\texttt{-4.833020050815520e-02}\\
\texttt{1.715159442482511e-01}\\
\texttt{9.711221038679012e-03}\\
\texttt{-7.835267507349471e-02}\\
\end{array}
\right],&
                                                    \left[
                                                    \begin{array}{c}
                                                      y_{0.85,2}\\
                                                      y_{0.85,222222}\\
                                                      y_{0.85,300000}\\
                                                      y_{0.85,400000}\\
                                                    \end{array}
    \right]&=&\left[\begin{array}{r}
\texttt{-4.108411686447251e-02}\\
\texttt{1.457851061784729e-01}\\
\texttt{8.251091450336577e-03}\\
\texttt{-6.660322024501109e-02}\\
\end{array}
\right]
  \end{array}
\end{equation*}
\newpart
%--------------------------
%    Problem 4 Part B
%--------------------------
\subsection*{(b)}
\lstinputlisting[style=Matlab-editor,basicstyle=\ttfamily\small]{powerm.m}

\newpage
\newpart
%--------------------------
%    Problem 4 Part C
%--------------------------
\subsection*{(c)}
The following code was used to compute the dominant eigenvalue $\lambda$, the iteration count $k$, the \textit{relative eigenvalue residual} $\epsilon$, and the corresponding eigenvector $x$:
\lstinputlisting[style=Matlab-editor,basicstyle=\ttfamily\small]{homework1_4_c.m}
with the output given by
\begin{eqnarray*}
  \begin{array}{rclcrcl}
    \left[\begin{array}{c}
            \lambda_1\\
            k_1\\
            \epsilon_1
          \end{array}
    \right]
 &=&\left[\begin{array}{r}
\texttt{9.999999999999998e-01}\\
\texttt{4.600000000000000e+01}\\
\texttt{8.881784197001246e-16}\\
\end{array}
\right]&&\left[\begin{array}{c}
                                             \lambda_2\\
                                             k_2\\
                                             \epsilon_2
                                           \end{array}
    \right]&=&\left[\begin{array}{r}
\texttt{9.999999999999998e-01}\\
\texttt{1.010000000000000e+02}\\
\texttt{6.661338147750936e-16}\\
\end{array}
\right]\\
    x&=&\left[\begin{array}{r}
\texttt{2.154770537656687e-01}\\
\texttt{2.386822749404331e-01}\\
\texttt{2.751476225007769e-01}\\
\texttt{3.633067440174039e-01}\\
\texttt{2.582616803066404e-01}\\
\texttt{2.300839117372837e-01}\\
\texttt{1.000000000000000e+00}\\
\texttt{2.386822749404331e-01}\\
\texttt{4.407956075831346e-01}\\
\texttt{3.247695017093132e-01}\\
\end{array}
\right]&&x&=&\left[\begin{array}{r}
\texttt{2.154770537656687e-01}\\
\texttt{2.386822749404331e-01}\\
\texttt{2.751476225007770e-01}\\
\texttt{3.633067440174039e-01}\\
\texttt{2.582616803066405e-01}\\
\texttt{2.300839117372837e-01}\\
\texttt{1.000000000000000e+00}\\
\texttt{2.386822749404331e-01}\\
\texttt{4.407956075831347e-01}\\
\texttt{3.247695017093132e-01}\\
\end{array}
\right].
  \end{array}
\end{eqnarray*}
Thus the ranking of the websites from most to least important is given by the row vectors:
\begin{eqnarray*}
  b_1&=&\left[\begin{array}{llllllllll}
\texttt{7}&\texttt{9}&\texttt{4}&\texttt{10}&\texttt{3}&\texttt{5}&\texttt{2}&\texttt{8}&\texttt{6}&\texttt{1}\\
\end{array}
\right].\\
  b_2&=&\left[\begin{array}{llllllllll}
\texttt{7}&\texttt{9}&\texttt{4}&\texttt{10}&\texttt{3}&\texttt{5}&\texttt{2}&\texttt{8}&\texttt{6}&\texttt{1}\\
\end{array}
\right].
\end{eqnarray*}

\newpage
\newpart
%--------------------------
%    Problem 4 Part D
%--------------------------
\subsection*{(d)}
The following code was used
\lstinputlisting[style=Matlab-editor,basicstyle=\ttfamily\small]{homework1_4_d.m}
where the function \texttt{poweram} is given by
\lstinputlisting[style=Matlab-editor,basicstyle=\ttfamily\small]{poweram.m}
For $x_1=e$ and $x_2=x_0$, the results are
\begin{equation*}
  \begin{array}{rclcrcl}
    \left[\begin{array}{c}
            \lambda_1\\
            k_1\\
            \varepsilon_1
          \end{array}
    \right]
 &=&\left[\begin{array}{r}
\texttt{1.000000895682723e+00}\\
\texttt{1.000000000000000e+04}\\
\texttt{1.097372320320294e-10}\\
\end{array}
\right]&&\left[\begin{array}{c}
                                             \lambda_2\\
                                             k_2\\
                                             \epsilon_2
                                           \end{array}
    \right]&=&\left[\begin{array}{r}
\texttt{8.896461299962105e-01}\\
\texttt{1.000000000000000e+04}\\
\texttt{2.085300260832659e-01}\\
\end{array}
\right]
  \end{array}
\end{equation*}
The convergence of the power method for this case is much slower than the $n=10$ case. The algorithm reached the maximum iterations instead of reaching the residual epsilon.
For $A_{0.85}^T$, the output is given by
\begin{equation*}
  \begin{array}{rclcrcl}
    \left[\begin{array}{c}
            \lambda_1\\
            k_1\\
            \varepsilon_1
          \end{array}
    \right]
 &=&\left[\begin{array}{r}
\texttt{9.999999999951747e-01}\\
\texttt{1.220000000000000e+02}\\
\texttt{9.277023593821179e-13}\\
\end{array}
\right]&&\left[\begin{array}{c}
                                             \lambda_2\\
                                             k_2\\
                                             \epsilon_2
                                           \end{array}
    \right]&=&\left[\begin{array}{r}
\texttt{1.000000000004541e+00}\\
\texttt{1.630000000000000e+02}\\
\texttt{9.168221737304508e-13}\\
\end{array}
\right]
  \end{array}
\end{equation*}
Thus the ranking of the top 10  websites from most to least important is given by the row vectors:
\begin{equation*}
  \begin{array}{r@{\hspace{0cm}}c@{\hspace{0cm}}l}
    b_1&=&\left[\begin{array}{llllllllll}
\texttt{629103}&\texttt{328995}&\texttt{176090}&\texttt{5397}&\texttt{609117}&\texttt{539881}&\texttt{63085}&\texttt{351516}&\texttt{529968}&\texttt{363332}\\
\end{array}
\right]\\
    b_2&=&\left[\begin{array}{llllllllll}
\texttt{629103}&\texttt{328995}&\texttt{176090}&\texttt{5397}&\texttt{609117}&\texttt{539881}&\texttt{63085}&\texttt{351516}&\texttt{529968}&\texttt{363332}\\
\end{array}
\right].
  \end{array}
\end{equation*}

\newquestion
%======================================================
%
%                    Problem 5
%
%======================================================
\section*{Problem 5}

\newpart
%--------------------------
%    Problem 5 Part A
%--------------------------
\subsection*{(a)}
The eigenvalues of $C$ can be found by \texttt{conj(fft(c'))} by the following formulas.
\begin{eqnarray*}
  \left[Fc^T\right]_j&=&\sum_{k=0}^{n-1}c_k\exp\left(\frac{-2\pi i}{n}jk\right),\quad\forall\,0\leq j\leq n-1\\
  \overline{\left[Fc^T\right]_j}&=&\sum_{k=0}^{n-1}c_k\exp\left(\frac{2\pi i}{n}jk\right),\quad\forall\,0\leq j\leq n-1
\end{eqnarray*}

\newpart
%--------------------------
%    Problem 5 Part B
%--------------------------
\subsection*{(b)}

Let 
\begin{equation*}
  \Lambda=\left[
    \begin{array}{cccc}
      \lambda_0 & 0 & \cdots & 0\\
      0 & \lambda_1 & \ddots & \vdots\\
      \vdots & \ddots & \ddots & 0\\
      0 & \cdots & 0 & \lambda_{n-1}
    \end{array}
  \right]
\end{equation*}
and $\texttt{lambda}=\texttt{conj(fft(c'))}$. Then $\Lambda Fx=\texttt{lambda.*fft(x)}$. Also,
\begin{eqnarray*}
  \left[\overline{F}x\right]_{j=0,1,\dots,n-1}&=&\left[\sum_{k=0}^{n-1}x_k\exp\left(\frac{2\pi i}{n}jk\right)\right]_{j,k=0,1,\dots,n-1}\\
                                              &=&\overline{\left[\sum_{k=0}^{n-1}\overline{x_k}\exp\left(\frac{-2\pi i}{n}jk\right)\right]}_{j=0,1,\dots,n-1}\\
                                              &=&\overline{\left[F\overline{x}\right]}_{j=0,1,\dots,n-1}\\
                                              &=&\texttt{conj(fft(conj(x)))}
\end{eqnarray*}
Therefore, $y=Cx=\frac{1}{n}\overline{F}\Lambda Fx=\texttt{conj(fft(conj(fft(c')).*fft(x))))/length(x)}$. Since $y$ is real, then $y=\texttt{fft(conj(fft(c')).*fft(x)))/length(x)}$. The following \MATLAB code was written to compare the two methods of multiplying a circulant matrix:
\lstinputlisting[style=Matlab-editor,basicstyle=\ttfamily\small]{homework1_5_b.m}
where the function \texttt{circulant} is given by
\lstinputlisting[style=Matlab-editor,basicstyle=\ttfamily\small]{circulant.m}
with the following output
\begin{eqnarray*}
  \texttt{C*x}&=&\left[\begin{array}{r}
\texttt{-1.600000000000000e+01}\\
\texttt{6.000000000000000e+00}\\
\texttt{-8.000000000000000e+00}\\
\texttt{6.000000000000000e+00}\\
\end{array}
\right]\\
  \texttt{circulant(c,x)}&=&\left[\begin{array}{r}
\texttt{-1.600000000000000e+01}\\
\texttt{6.000000000000000e+00}\\
\texttt{-8.000000000000000e+00}\\
\texttt{6.000000000000000e+00}\\
\end{array}
\right].
\end{eqnarray*}

\newpart
%--------------------------
%    Problem 5 Part C
%--------------------------
\subsection*{(c)}
If we extend the Toeplitz matrix such that $t_j=t_{j+1-2n},t_{-j}=t_{2n-1-j}$ for $n< j\leq 2n-1$, then by definition we have a unique, circulant matrix such that $c_j:=t_j$.

\newpage
\newpart
%--------------------------
%    Problem 5 Part D
%--------------------------
\subsection*{(d)}
The following \MATLAB function uses part (c) and part (b) to compute $y=Tx$:
\lstinputlisting[style=Matlab-editor,basicstyle=\ttfamily\small]{toeplitz.m}

\newpart
%--------------------------
%    Problem 5 Part E
%--------------------------
\subsection*{(e)}
The following \MATLAB code was written to compare the two methods of multiplying a Toeplitz matrix
\lstinputlisting[style=Matlab-editor,basicstyle=\ttfamily\small]{homework1_5_e.m}
and has the following output
\begin{eqnarray*}
  \texttt{T*x}&=&\left[\begin{array}{r}
\texttt{-1.200000000000000e+01}\\
\texttt{1.800000000000000e+01}\\
\texttt{5.000000000000000e+00}\\
\texttt{-7.000000000000000e+00}\\
\end{array}
\right]\\
  \texttt{toeplitz(t,x)}&=&\left[\begin{array}{r}
\texttt{-1.200000000000000e+01}\\
\texttt{1.800000000000000e+01}\\
\texttt{5.000000000000000e+00}\\
\texttt{-7.000000000000001e+00}\\
\end{array}
\right].
\end{eqnarray*}
For the $n=500000$ case, the code is
\lstinputlisting[style=Matlab-editor,basicstyle=\ttfamily\small]{homework1_5_e2.m}
and has the output
\begin{equation*}
  \left[\begin{array}{c}
          y_1\\
          y_{100000}\\
          y_{200000}\\
          y_{300000}\\
          y_{400000}\\
          y_{500000}\\
          \textstyle \sum_{j=1}^ny_j\\
          \left\Vert y\right\Vert_2
        \end{array}
      \right]=\left[\begin{array}{r}
\texttt{4.743400116337570e+01-1.897571643350681e-14i}\\
\texttt{-2.044536337994736e+02+3.463356617877540e-14i}\\
\texttt{9.854729389973313e+01-2.893191580611515e-14i}\\
\texttt{-2.273143080393689e+01+4.576858382960006e-14i}\\
\texttt{4.136625635601638e+02-5.480763011145771e-14i}\\
\texttt{2.524304751530100e+02+9.171369375379089e-14i}\\
\texttt{1.030143367304304e+05+7.836054370961132e-13i}\\
\texttt{1.669299439051479e+05}\\
\end{array}
\right].
\end{equation*}

\end{document}
%================================================================
%================================================================
%
%                           Templates
%
%================================================================
%================================================================


%----------------------------------------------------------------
%----------------------------------------------------------------


\newquestion
%======================================================
%
%                    Problem n
%
%======================================================
\section*{Problem n}

\newpart
%--------------------------
%    Problem n Part A
%--------------------------
\subsection*{(a)}

\newpart
%--------------------------
%    Problem n Part B
%--------------------------
\subsection*{(b)}


%----------------------------------------------------------------
%----------------------------------------------------------------


%======================================================
%
%               Appendix: Problem n
%
%======================================================
%--------------------------
%  Appendix: P n Part A
%--------------------------
\subsection*{Problem n Part A}

\newpage
%--------------------------
%  Appendix: P n Part B
%--------------------------
\subsection*{Problem n Part B}


%----------------------------------------------------------------
%----------------------------------------------------------------